\documentclass{article}
\usepackage{graphicx}
\usepackage{amsmath}
\usepackage{booktabs}
\usepackage{cite}
\usepackage{geometry}
\usepackage{microtype}
\usepackage[colorlinks=true,linkcolor=blue,citecolor=blue,urlcolor=blue]{hyperref}
\usepackage{silence}
\WarningFilter{latex}{Command \showhyphens has changed}
\geometry{left=3cm,right=3cm,top=3cm,bottom=3cm}


\title{Replacing Dark Energy and Unifying Large-Scale Anomalies: \\ The Progenitor Node Model within a Virialized Meta-Structure}
\author{Martin Gamsby}
\date{February 2026}

\begin{document}

\maketitle

\begin{abstract}
Standard $\Lambda$CDM cosmology \cite{Planck2020} attributes the observed late-time cosmic acceleration to an intrinsic, theoretically unexplained vacuum energy known as Dark Energy.
While this parameter provides a fit to observational data, it suffers from severe theoretical pathologies, most notably the fine-tuning problem \cite{Weinberg1989}—requiring a cancellation of vacuum energy terms across 120 orders of magnitude—and the coincidence problem \cite{Steinhardt1999}.
This paper proposes a radical alternative: a geometric framework termed the \textbf{External-Node Model}.
We posit that the observable universe—modeled as a finite Friedmann-Robertson-Walker (FRW) region—originates from the destabilization of a Progenitor Node situated within a larger, static, and virialized meta-structure composed of Hyper-Massive External Attractors (HMEAs).

In this framework, cosmic acceleration arises not from a repulsive "push" from the vacuum, but from the tidal gravitational "pull" of the surrounding grid as the bubble's radius expands toward the effective capture radius of neighbor nodes.
Through N-body simulations covering the late-universe expansion era ($t=5.8 \rightarrow 13.8$ Gyr), we demonstrate that multiple grid configurations across a wide mass range ($M = 92$-$9000 \times M_{obs}$, $S = 15$-$38$ Gpc) reproduce $\Lambda$CDM expansion dynamics with $R^2 > 0.99$ for size evolution and $R^2 > 0.95$ for expansion rate, without invoking exotic fluids or modified gravity.
Beyond the expansion match, the same discrete lattice geometry---with no additional free parameters---generates predictions consistent with three independent large-scale anomalies: the Hubble Tension ($\Delta H_0/H_0 \approx 5$-$11\%$ \cite{Riess2022}), the CMB ``Axis of Evil'' \cite{Land2005}, and large-scale ``dark flow'' \cite{Watkins2023}.
All three are predicted to share a common preferred direction toward the nearest HMEA node, providing a unified geometric origin for observations otherwise treated as unrelated statistical flukes.

While this work presents a simplified toy model that does not yet address early universe physics, it demonstrates that dark energy is not inevitable;
classical gravitational alternatives exist that simultaneously address the expansion history and multiple observed large-scale anomalies.
This proof-of-concept challenges the assumption that the Cosmological Principle holds at super-horizon scales.
\end{abstract}

\clearpage

\tableofcontents

\clearpage

\section{Introduction}
\label{sec:intro}

The discovery of the accelerating expansion of the universe remains the most significant theoretical challenge in modern cosmology \cite{Riess1998, Perlmutter1999}.
To reconcile this acceleration with General Relativity, the standard $\Lambda$CDM model reintroduces the Cosmological Constant ($\Lambda$), interpreted as a constant energy density inherent to the vacuum.
Although $\Lambda$ fits Type Ia supernovae and Cosmic Microwave Background (CMB) data with remarkable precision ($\Omega_\Lambda \approx 0.7$), it suffers from profound theoretical deficiencies suggesting it may be a mathematical placeholder rather than a physical explanation.

First is the \textbf{Fine-Tuning Problem} \cite{Weinberg1989}: predictions of vacuum energy density derived from Quantum Field Theory exceed the observed value of $\Lambda$ by approximately 120 orders of magnitude.
To align theory with observation requires an inexplicable cancellation of terms to an unnatural precision.

Second is the \textbf{Coincidence Problem} \cite{Steinhardt1999}: why do we live in the unique cosmological epoch where dark energy and matter densities are comparable ($\Omega_\Lambda \sim \Omega_m$)?
In the standard model, this appears to be a statistical fluke.

This paper explores a classical solution based on embedding geometry, eschewing the need for quantum vacuum energy entirely.
We abandon the assumption that the FRW metric describes the totality of spacetime.
Instead, we investigate the dynamics of a finite FRW region expanding within a 3D Euclidean meta-space populated by discrete, massive gravitational anchors.

We propose that "Dark Energy" is a local misinterpretation of global boundary conditions.
Specifically, we replace the \textbf{intrinsic repulsion} of the vacuum with the \textbf{extrinsic attraction} of a surrounding mass distribution.
We term this the \textbf{External-Node Model}. By doing so, we naturally resolve the fine-tuning problem—gravity is weak on large scales, so the forces are naturally small—and provide a unified explanation for persistent tensions in astrophysical data.
A companion proposal, the \textbf{Progenitor Hypothesis} (Section~\ref{sec:progenitor}), resolves the isotropy fine-tuning problem by positing that the Big Bang originated from a destabilized node within the meta-structure, inheriting the symmetry of its equilibrium position.

The idea that trans-observable structure could influence cosmological dynamics is not entirely new.
Mersini-Houghton and Holman \cite{MersiniHoughton2009} proposed that entanglement with neighboring domains in the string landscape could produce large-scale bulk flows and CMB anomalies, while several authors have noted that matter beyond the particle horizon could source coherent motions \cite{Kashlinsky2008}.
However, prior proposals have typically addressed individual anomalies in isolation using purpose-built mechanisms.
Our framework differs in a critical respect: a \textit{single} discrete lattice geometry, motivated by the requirement to reproduce cosmic acceleration, \textit{simultaneously} and \textit{without additional parameters} generates predictions consistent with the Hubble Tension, the CMB ``Axis of Evil,'' and large-scale bulk flows (Section~\ref{sec:predictions}).
This unification under one geometric structure is, to our knowledge, novel.

Critically, this work demonstrates that dark energy is not inevitable: classical gravitational mechanisms can reproduce the observed acceleration without requiring new physics.
While the model presented here is a simplified proof-of-concept, it challenges fundamental assumptions about the Cosmological Principle at super-horizon scales and may inspire more complete theories of cosmic dynamics.

\section{The Physical Model}
\label{sec:model}

\subsection{The Virialized Grid Topology}
\label{sec:grid}
Standard cosmology assumes the Cosmological Principle—homogeneity and isotropy—holds indefinitely.
We propose this is valid only within our local "bubble."
Outside our horizon, we model the meta-universe as a \textbf{Virialized Meta-Structure}.

This structure is conceptually similar to a super-scaled cosmic web governed by virial dynamics.
This implies that the arrangement of mass concentrations is stable over vast timescales, balanced by mutual gravitation, but irregular in detail—much like the distribution of galaxies within a relaxed cluster.

These concentrations, termed \textbf{Hyper-Massive External Attractors (HMEAs)}, are compact objects—likely black holes far older than our universe's 13.8 Gyr—with masses exceeding the total mass of our observable universe ($M_{ext} \gg M_{obs}$) and separated by voids ($S \gg R_{H}$).
Our universe exists within one of these voids, expanding outward toward the surrounding nodes.
The "grid" serves as the gravitational boundary condition for our spacetime bubble.

\subsection{The Progenitor Hypothesis}
\label{sec:progenitor}
A common objection to "bubble" cosmologies is the Copernican fine-tuning problem: why is our universe located precisely in the center of the void, such that expansion appears isotropic?

We resolve this via the \textbf{Progenitor Hypothesis}. We posit that the Big Bang was not a creation event \textit{ex nihilo} in random empty space, but a phase transition event of a specific object.

\begin{enumerate}
\item \textbf{The Origin}: Our universe originates from a \textbf{Progenitor Node} that was previously a constituent member of the Virialized Grid.
It sat at a gravitational equilibrium point relative to its neighbors, balanced by the forces of the surrounding lattice.
\item \textbf{The Event}: This node underwent a destabilization event, perhaps triggered by accumulating critical mass, internal quantum instability (such as Hawking radiation thresholds), or a collision with another body.
It transitioned from a bound object to a rapidly expanding cloud of matter and radiation (a Big Bang, our Big Bang).

\item \textbf{Isotropy}: Because the Progenitor Node was gravitationally locked in equilibrium with its neighbors prior to the event, the resulting expansion originates from a point of pre-existing dynamic stability.
The matter expands symmetrically into the potential wells of the surrounding meta-structure, naturally preserving isotropy without requiring fine-tuning of initial conditions.
\end{enumerate}

\section{Dynamics and Quantitative Analysis}
\label{sec:dynamics}
In this framework, the scale factor $a(t)$ of the universe is governed by the competitive interplay between internal self-gravity (which causes deceleration, as in the standard matter-dominated era) and external tidal gravity (which mimics acceleration).

\subsection{The Tidal Acceleration Mechanism}
\label{sec:tidal}
In standard Newtonian cosmology, a spherical shell of matter decelerates solely due to internal mass $M_{int}$ enclosed by the shell.
This is a consequence of Birkhoff's Theorem \cite{Birkhoff1923}, which states that external spherical shells exert no net force on the interior.
However, our model violates the conditions of Birkhoff's Theorem: the external distribution is not a continuous spherical shell, but a discrete grid of massive points.

As the bubble expands, it experiences a tidal force—a differential gravitational pull that increases with distance from the center.
Consider a test galaxy at the edge of the expanding bubble.
The net force from the surrounding isotropic grid is zero at the exact center ($R=0$).
As the galaxy moves outward, however, it climbs the potential gradient of the nearest HMEA.

The tidal acceleration $a_{tidal}$ caused by an external mass $M_{ext}$ at distance $S$ scales as the derivative of the gravitational force with respect to position.
Differentiating the Newtonian force equation $F = -GM/r^2$ yields the tidal term:

\begin{equation}
a_{tidal} \approx \frac{G M_{ext}}{(S - R)^2} - \frac{G M_{ext}}{S^2}
\label{eq:tidal-exact}
\end{equation}

In the limit where the bubble radius is small compared to the grid spacing ($R \ll S$), we can Taylor expand Equation~\ref{eq:tidal-exact}, yielding a linear dependence on $R$:

\begin{equation}
a_{tidal} \approx \frac{2 G M_{ext}}{S^3} R
\label{eq:tidal-linear}
\end{equation}

We emphasize that this derivation considers only the nearest node.
This is physically justified: in a virialized meta-structure, the gravitational forces from all surrounding nodes cancel at the equilibrium point ($R = 0$) by definition of virial balance.
As the bubble expands, the tidal contribution from each node scales as $\sim GM/d^3$ where $d$ is the node distance; the nearest face node at distance $S$ therefore dominates over edge nodes at $S\sqrt{2}$ (contributing $\sim 35\%$ as much) and corner nodes at $S\sqrt{3}$ ($\sim 19\%$).
The N-body simulation (Section~\ref{sec:numerical}) uses all 26 nodes; the single-node derivation here serves as an order-of-magnitude estimate of the mechanism's viability.

This linear dependence on $R$ is crucial.
In the Friedmann equations, the repulsive acceleration due to a Cosmological Constant is given by $H_0^2 \Omega_\Lambda R$.
Comparing this with Equation~\ref{eq:tidal-linear}, both tidal acceleration and Dark Energy acceleration scale linearly with $R$; a uniform external tidal field is therefore mathematically indistinguishable from an intrinsic cosmological constant in the non-relativistic limit.
The "Dark Energy" we observe is simply the tidal tension of the meta-grid pulling the universe apart.

\subsection{Estimation of Required Node Mass and Distance}
\label{sec:estimation}
To determine if this hypothesis is physically plausible, we solve for the grid parameters required to match current observations.
We equate the tidal acceleration (Equation~\ref{eq:tidal-linear}) to the observed Dark Energy acceleration:

\begin{equation}
H_0^2 \Omega_\Lambda \approx \frac{2 G M_{ext}}{S^3}
\label{eq:matching}
\end{equation}

We rearrange this to solve for the necessary grid spacing distance $S$:

\begin{equation}
S \approx \left( \frac{2 G M_{ext}}{H_0^2 \Omega_\Lambda} \right)^{1/3}
\end{equation}

We must posit a mass for the HMEA.
We assume $M_{ext} = 5 \times 10^{55} \text{ kg}$, which is approximately 500 times the estimated mass of the observable universe ($M_{obs} \approx 10^{53} \text{ kg}$).
Using the current Hubble constant $H_0 \approx 70 \text{ km/s/Mpc} \approx 2.3 \times 10^{-18} \text{ s}^{-1}$ and $\Omega_\Lambda \approx 0.7$:

\begin{equation}
S \approx \left( \frac{2 \times 6.67 \times 10^{-11} \cdot 5 \times 10^{55}}{3.7 \times 10^{-36}} \right)^{1/3} \approx 1.2 \times 10^{27} \text{ meters}
\end{equation}

Converting this result from meters to parsecs ($1 \text{ Mpc} \approx 3.08 \times 10^{22} \text{ m}$) yields:

\begin{equation}
S \approx 39 \text{ Gigaparsecs (Gpc)}
\end{equation}

\paragraph{Result:} A network of nodes with masses of $\sim 10^{55}$ kg spaced $\sim 40$ Gpc apart provides the correct magnitude of gravitational "lift" to reproduce the observed $\Lambda$.
This distance is significant because it places the attractors well outside the current particle horizon of the universe ($\sim 14$ Gpc), explaining why they are not immediately obvious in sky surveys—we are fundamentally causally disconnected from their light, but not their gravity.

\paragraph{Analytical vs. Numerical Discrepancy:}
While Equation~\ref{eq:matching} predicts $S \approx 39$ Gpc for $M_{ext} \approx 500 \times M_{obs}$, systematic numerical exploration reveals optimal matches across a broad parameter space ($S = 15$-$70$ Gpc), with balanced-optimization configurations clustering at $S = 15$-$38$ Gpc.
The broad range of viable $S$ values demonstrates that the mechanism is robust rather than fine-tuned to the analytical target.

\section{Numerical Validation}
\label{sec:numerical}

\subsection{N-Body Simulation Framework}
\label{sec:framework}
To rigorously test whether the external-node mechanism can reproduce $\Lambda$CDM expansion dynamics, we implemented a computational N-body simulation.
The simulation models our observable universe as a collection of test particles expanding under the combined influence of internal self-gravity and external tidal forces from the HMEA grid.

\paragraph{Grid Configuration:}
We model the meta-structure as a 3×3×3 cubic lattice with 26 external nodes surrounding our observable universe.
The central position represents our universe (no node present), with HMEAs positioned at integer multiples of the grid spacing $S$.
This topology naturally enforces approximate isotropy while accounting for the discrete nature of the gravitational sources.
We emphasize that this symmetric grid is a \textbf{simplified representation} for computational tractability—a real virialized meta-structure would contain nodes of varying masses at different distances with irregular spacing.
However, nodes at distances significantly greater than the nearest neighbors contribute negligibly to the tidal force, making the 3×3×3 approximation sufficient for validating the core mechanism.
The purpose of this toy model is to test whether the concept is viable, not to claim exact correspondence with reality.

\paragraph{Simulation Parameters:}
The simulation evolves 2000 test particles over an 8 Gyr period (from cosmic time $t = 5.8$ Gyr to $t = 13.8$ Gyr), covering the era of late-universe expansion and observed cosmic acceleration.
We begin at $t = 5.8$ Gyr rather than the Big Bang because this toy model explicitly focuses on late-time acceleration—early universe physics are outside our current scope (Section~\ref{sec:limitations}).
Initial conditions are set to match $\Lambda$CDM predictions at $t = 5.8$ Gyr.

\paragraph{Parameter Exploration Methodology:}
We conducted systematic parameter sweeps using an adaptive linear search algorithm with intelligent step-skipping to efficiently explore the $(M_{ext}, S)$ parameter space.
The $\Lambda$CDM baseline is computed analytically via exact Friedmann solution at N-body snapshot times, eliminating interpolation artifacts.
Each configuration is validated using the $R^2$ (coefficient of determination) metric, which measures the fraction of $\Lambda$CDM variance explained by the External-Node model.
Quality checks include:
\begin{enumerate}
\item Matter-only simulations must never exceed $\Lambda$CDM expansion at any timestep (physics constraint),
\item Center-of-mass drift monitoring to ensure grid symmetry, and
\item Runaway particle detection to flag numerical instability.
\end{enumerate}

\paragraph{Numerical Robustness:}
Results are validated across multiple random seeds; we report the worst-performing seed for each configuration to provide conservative estimates.
Convergence with respect to particle count shows that increasing $N$ beyond $\sim$2000 smooths statistical fluctuations but does not significantly change the $R^2$ metrics.
At low particle counts ($N \lesssim 500$), individual particles can represent an unphysically large mass fraction ($\gtrsim 1/50$ of $M_{obs}$), occasionally causing a single particle to be ejected from the cluster by close gravitational encounters—an artifact of insufficient resolution rather than a physical instability.
Similarly, too few timesteps can under-resolve close encounters.
Above the resolution threshold, results are stable: the expansion dynamics are a collective property of the particle cloud, insensitive to the exact number of tracers.

\begin{figure}[htbp]
    \centering
    \includegraphics[width=1.0\linewidth]{sim_plots_4f97dec_2000p_5.8-13.8Gyr_9000.0M_1centerM_38.0S_300steps_123seed_0.0rnd_Autod.png}
    \caption{Comparison of $\Lambda$CDM (blue solid), External-Node model with M=9000$\times$M$_{obs}$, S=38 Gpc (red dashed), and matter-only (green dotted) showing: (top left) scale factor evolution, (top right) Hubble parameter evolution with realistic present-day value $H_0 \approx 70$ km/s/Mpc, (bottom left) ratio of physical sizes, and (bottom right) physical universe size versus node distance.
The external-node model tracks $\Lambda$CDM closely while matter-only progressively diverges during the acceleration era.
This balanced-optimization configuration achieves 98.02\% endpoint match with $R^2 = 0.9954$ for size evolution and $R^2 = 0.9630$ for expansion rate.}
\label{fig:primary}
\end{figure}

\begin{figure}[htbp]
    \centering
    \includegraphics[width=0.87\linewidth]{sim_plots_4f97dec_2000p_5.8-13.8Gyr_875.0M_1centerM_24.0S_300steps_123seed_0.0rnd_Autod.png}
    \includegraphics[width=0.87\linewidth]{sim_plots_4f97dec_2000p_5.8-13.8Gyr_92.0M_1centerM_15.0S_300steps_123seed_0.0rnd_Autod.png}
    \caption{Additional balanced-optimization configurations demonstrating mechanism robustness: (top) M=875$\times$M$_{obs}$, S=24 Gpc, and (bottom) M=92$\times$M$_{obs}$, S=15 Gpc.
In all panels, $\Lambda$CDM is blue solid, External-Node is red dashed, and matter-only is green dotted.
Multiple solutions across a wide mass range all reproduce $\Lambda$CDM expansion dynamics with $R^2 > 0.99$ for size and $R^2 > 0.95$ for expansion rate, while matter-only consistently diverges.}
\label{fig:additional}
\end{figure}

\subsection{Parameter Space Exploration and Results}
\label{sec:results}

Through systematic parameter exploration, we identified \textbf{multiple parameter combinations} that reproduce $\Lambda$CDM expansion with high fidelity (Table~\ref{tab:results}).
There is no single "optimal" configuration. A family of solutions exists across a broad parameter range, demonstrating the robustness of the external-node mechanism.

We report results using two complementary optimization strategies:

\textbf{Balanced optimization} weights both size evolution and expansion rate, selecting configurations that match the full expansion dynamics.
\textbf{Size-only optimization} maximizes the size curve $R^2$ alone, which can achieve higher size fidelity at the cost of expansion rate agreement (see Section~\ref{sec:tradeoffs}).

\begin{table}[htbp]
\centering
\begin{tabular}{lcccc}
\toprule
\textbf{M} & \textbf{S [Gpc]} & \textbf{Endpoint} & \textbf{Size $R^2$} & \textbf{Expansion $R^2$} \\
\midrule
\multicolumn{5}{c}{\textit{Balanced optimization}} \\
$9000 \times M_{obs}$ & 38 & 98.02\% & 0.9954 & 0.9630 \\
$875 \times M_{obs}$ & 24 & 97.92\% & 0.9951 & 0.9603 \\
$92 \times M_{obs}$ & 15 & 98.67\% & 0.9966 & 0.9565 \\
\midrule
\multicolumn{5}{c}{\textit{Size-only optimization}} \\
$800 \times M_{obs}$ & 22 & 99.85\% & 0.9991 & (not optimized) \\
\midrule
\multicolumn{5}{c}{\textit{Baseline}} \\
Matter-only & --- & 96.06\% & 0.9890 & 0.8350 \\
\bottomrule
\end{tabular}
\caption{Parameter configurations and fit quality. Balanced optimization weights both size and expansion rate; size-only optimization maximizes size $R^2$ alone. Matter-only baseline uses no dark energy and no external nodes.}
\label{tab:results}
\end{table}

This multiplicity demonstrates that the mechanism is not fine-tuned to a singular point in parameter space.
The mass range spans over two orders of magnitude ($92$-$9000 \times M_{obs}$), yet all configurations reproduce $\Lambda$CDM dynamics with $R^2 > 0.99$ for size and $R^2 > 0.95$ for expansion rate---substantially exceeding the matter-only baseline ($R^2_{rate} = 0.84$; Figure~\ref{fig:additional}).
Close configurations ($S = 15$ Gpc, $R/S \approx 0.48$ today) may predict hyper-acceleration on timescales of tens of billions of years as the universe approaches the node boundary, offering testable predictions for deep-time cosmology.

\paragraph{$R^2$ Metric and Statistical Rigor:}
We employ the coefficient of determination ($R^2$) to quantify agreement with $\Lambda$CDM:
\[
R^2 = 1 - \frac{\sum (y_{model} - y_{\Lambda CDM})^2}{\sum (y_{\Lambda CDM} - \bar{y}_{\Lambda CDM})^2}
\]
where $R^2 = 1$ indicates perfect fit, $R^2 = 0$ means the model performs no better than the mean, and $R^2 < 0$ indicates the model is worse than a constant baseline.
We compute $R^2$ for both size evolution and expansion rate over the full 8 Gyr simulation period.

\subsection{Optimization Tradeoffs: Size vs. Expansion Rate}
\label{sec:tradeoffs}

The parameter sweep reveals an important distinction between two complementary measures of fit.
\textbf{Size evolution} $R^2$ measures how well the simulated universe radius tracks the $\Lambda$CDM curve---an integrated quantity that is inherently smooth.
\textbf{Expansion rate} $R^2$ measures how well the simulated Hubble-like parameter $H(t)$ tracks the Friedmann solution---a derivative quantity that is more physically demanding to match.

Optimizing solely for size $R^2$ yields configurations such as M = 800 $\times$ M$_{obs}$, S = 22 Gpc, which achieves $R^2_{size} = 0.9991$ with 99.85\% endpoint accuracy.
However, this comes at the cost of expansion rate fidelity: the simulated $H(t)$ follows a qualitatively different path that happens to integrate to nearly the same size curve.
This is expected---many velocity profiles can produce the same displacement over a fixed interval.

A balanced optimization that weights both size and expansion rate identifies a family of configurations with $R^2_{size} > 0.99$ and $R^2_{rate} > 0.95$ across a wide mass range (Section~\ref{sec:results}).
For comparison, the matter-only simulation achieves $R^2_{rate} \approx 0.84$, confirming that external nodes substantially improve the expansion rate match---not merely the integrated size.

Importantly, the expansion rate comparison is inherently approximate.
We compare the isotropic RMS expansion rate of our N-body cloud to the Friedmann equation's $H(t)$, but the external-node model predicts anisotropic expansion (Section~\ref{sec:dipole}).
A spherically-averaged measurement necessarily smooths over directional differences.
Moreover, the target curve itself is uncertain: the Hubble tension ($\sim$8.6\% discrepancy between early- and late-universe measurements of $H_0$) means the ``correct'' $H(t)$ is not settled.
A toy model that reproduces the overall expansion history to $R^2_{size} > 0.99$ while matching the expansion rate to $R^2_{rate} > 0.95$ demonstrates the viability of the mechanism; the remaining discrepancy may partly reflect the model's own predicted anisotropy.

\paragraph{Quantitative Agreement:}
Using the balanced-optimization configuration $M_{ext} = 9000 \times M_{obs}$, $S = 38$ Gpc as the primary example (Figure~\ref{fig:primary}):
\begin{itemize}
    \item $\Lambda$CDM final size: 14.50 Gpc
    \item External-node final size: 14.21 Gpc
    \item Endpoint match: 98.02\%
    \item Size $R^2$: 0.9954
    \item Expansion rate $R^2$: 0.9630
\end{itemize}

The model tracks $\Lambda$CDM expansion dynamics throughout the 8 Gyr evolution, achieving strong agreement in both size evolution and expansion rate.
Size-only optimization can push the size $R^2$ to 0.9991 (endpoint 99.85\%), but this tradeoff is discussed in Section~\ref{sec:tradeoffs}.

\paragraph{Hubble Parameter Evolution:}
The time-dependent Hubble parameter $H(t)$ in our model exhibits strong quantitative agreement with the Friedmann solution ($R^2_{rate} > 0.95$ for balanced configurations):
\begin{itemize}
    \item Present-day value ($t \approx 13.8$ Gyr): $H \approx 70$ km/s/Mpc
    \item Characteristic rise-peak-decline signature matching accelerated expansion
    \item Peak occurs near the present epoch, consistent with the transition to dark energy domination
    \item Expansion rate $R^2 = 0.96$ versus matter-only $R^2 = 0.84$, confirming genuine acceleration dynamics
\end{itemize}

Critically, the model reproduces realistic Hubble parameter values without requiring adjustment—the gravitational forces from the external grid naturally generate the observed magnitude of cosmic acceleration.

\subsection{Matter-Only Comparison: Distinguishing Endpoint from Dynamics}
\label{sec:matter-only}

A critical validation is comparing the external-node model against a matter-only cosmology (no dark energy, no external nodes: pure gravitational deceleration).
This comparison exposes a subtle but crucial distinction: \textbf{endpoint match versus dynamics match}.

\paragraph{Matter-Only Performance:}
The matter-only model (evolving under internal self-gravity alone) produces decent endpoint agreement:
\begin{itemize}
    \item Final size: 13.93 Gpc (vs. 14.50 Gpc $\Lambda$CDM)
    \item Endpoint match: 96.06\%
    \item Size $R^2$: 0.9890
    \item Expansion rate $R^2$: 0.8350
\end{itemize}

The endpoint match appears respectable ($\sim$96\%), but the expansion rate $R^2$ reveals the underlying problem: the matter-only trajectory follows a qualitatively different expansion profile, progressively diverging from $\Lambda$CDM during the acceleration era.

\paragraph{External-Node vs Matter-Only:}
Comparing the balanced-optimization external-node configuration ($M_{ext} = 9000 \times M_{obs}$, $S = 38$ Gpc) to matter-only (Table~\ref{tab:comparison}):

\begin{table}[htbp]
\centering
\begin{tabular}{lccc}
\toprule
\textbf{Model} & \textbf{Endpoint} & \textbf{Size $R^2$} & \textbf{Expansion $R^2$} \\
\midrule
$\Lambda$CDM (baseline) & 100\% & 1.0000 & 1.0000 \\
External-Node & 98.02\% & 0.9954 & 0.9630 \\
Matter-only & 96.06\% & 0.9890 & 0.8350 \\
\bottomrule
\end{tabular}
\caption{External-Node vs matter-only comparison for $M_{ext} = 9000 \times M_{obs}$, $S = 38$ Gpc. The expansion rate $R^2$ discriminates between genuine acceleration dynamics and coincidental endpoint agreement.}
\label{tab:comparison}
\end{table}

The external-node model achieves $R^2_{size} = 0.995$ and $R^2_{rate} = 0.963$, demonstrating it replicates the acceleration \textit{dynamics}---both size and expansion rate---throughout the trajectory.
The expansion rate comparison is particularly revealing: external nodes achieve $R^2_{rate} = 0.96$ versus matter-only's 0.84, a substantial improvement that confirms the external-node mechanism provides genuine acceleration dynamics, not merely a coincidental endpoint match.

\paragraph{Significance:}
This comparison proves that external nodes provide an acceleration mechanism comparable to dark energy, not merely a fortuitous final size.
The size $R^2$ serves as a discriminator for spurious matches: matter-only cannot replicate acceleration dynamics across 8 Gyr of late-time evolution, even if it reaches a reasonably similar endpoint.

\subsection{Physical Interpretation}
\label{sec:interpretation}

These results demonstrate several key points:

\begin{enumerate}
    \item \textbf{Mechanism Validation}: External gravitational sources can produce accelerated expansion matching $\Lambda$CDM with $R^2_{size} > 0.99$ and $R^2_{rate} > 0.95$ using only classical Newtonian gravity. Size-only optimization can push $R^2_{size}$ to 0.9991, demonstrating the flexibility of the parameter space.
\item \textbf{Parameter Economy}: The model's primary free parameters are $M_{ext}$ and $S$, which jointly determine the tidal acceleration.
The simulation additionally requires grid topology (fixed at 3$\times$3$\times$3) and an initial velocity calibration derived from the starting epoch, but neither is tuned to the expansion data---the grid is the minimal symmetric configuration, and the calibration ensures the simulation starts on the $\Lambda$CDM trajectory at $t = 5.8$ Gyr rather than biasing the outcome.
\item \textbf{Observational Consistency}: The Hubble parameter evolution matches current measurements, including the approximate value of $H_0 \sim 70$ km/s/Mpc at the present epoch, with $R^2_{rate} = 0.96$ for balanced configurations.
\item \textbf{Proof of Concept}: While this is a simplified toy model—assuming homogeneous expansion and neglecting density perturbations—it establishes that purely gravitational mechanisms can account for observed cosmic acceleration without invoking vacuum energy or modified gravity.
\item \textbf{Matter-Only Falsification}: The matter-only model's significantly lower expansion rate $R^2$ (0.84 vs 0.96 for external nodes) and size $R^2$ (0.989 vs 0.995) demonstrates that deceleration cannot mimic acceleration dynamics, even if endpoints approximately align.
This validates that external nodes provide a genuine acceleration mechanism.
\end{enumerate}

\section{Scope and Limitations of the Toy Model}
\label{sec:scope}

This work presents a \textbf{proof-of-concept toy model}, not a complete cosmological theory.
We explicitly acknowledge the following limitations and boundaries of our current framework:

\subsection{What the Model Addresses}
\label{sec:addresses}
The external-node model successfully demonstrates:
\begin{itemize}
    \item Late-time cosmic acceleration over 8 Gyr ($t = 5.8 \rightarrow 13.8$ Gyr)
    \item High-fidelity reproduction of $\Lambda$CDM expansion dynamics (Section~\ref{sec:results})
    \item A classical gravitational mechanism that mimics dark energy acceleration dynamics, not just endpoints
    \item Resolution of the fine-tuning problem (gravity is naturally weak)
    \item A natural explanation for the timing of acceleration (related to geometric scales)
    \item Multiple parameter solutions across two orders of magnitude in mass, demonstrating mechanism robustness
\end{itemize}

\subsection{What the Model Does Not Yet Address}
\label{sec:limitations}
Several critical aspects of cosmology remain outside the scope of this work:
\begin{itemize}
    \item \textbf{Early Universe Physics}: The model does not address inflation, baryogenesis, or primordial nucleosynthesis. The Progenitor destabilization event is posited but not derived from fundamental physics.
\item \textbf{CMB Power Spectrum}: We have not calculated how the external grid affects acoustic oscillations, the distance to last scattering, or the detailed structure of CMB anisotropies. Compatibility with Planck observations requires future investigation \cite{Planck2020}.
    \item \textbf{Structure Formation}: The toy model assumes homogeneous expansion. Density perturbations, hierarchical galaxy formation, and the growth of large-scale structure are not included.
\item \textbf{Baryon Acoustic Oscillations}: Standard ruler measurements from BAO \cite{Eisenstein2005} provide independent constraints on dark energy that we have not yet tested against our model.
\item \textbf{General Relativity}: Our analysis uses Newtonian gravity in a flat Euclidean embedding space.
A complete theory would require:
    \begin{itemize}
        \item Full relativistic treatment of the meta-structure's metric
        \item Junction conditions at the bubble boundary
        \item Proper relativistic formulation of tidal forces in curved spacetime
        \item Reconciliation with the Cosmological Principle's domain of validity
    \end{itemize}
    \item \textbf{Trans-Observable Structure}: The existence, observability, and detailed properties of HMEAs and the meta-structure remain speculative. We cannot yet specify how such a structure could form or be verified.
\end{itemize}

\subsection{Philosophical Stance}
\label{sec:philosophy}
We do not claim this model represents physical reality.
Rather, we demonstrate that:
\begin{enumerate}
    \item \textbf{Dark energy is not inevitable}—classical gravitational alternatives exist
    \item \textbf{Fundamental assumptions may break down}—the Cosmological Principle might not hold at super-horizon scales
    \item \textbf{Alternative mechanisms are viable}—external structure could explain acceleration
    \item \textbf{New research directions open}—even if ultimately superseded, this approach may inspire more complete theories
\end{enumerate}

The value of this work lies not in claiming to solve cosmology, but in demonstrating that well-motivated alternatives to vacuum energy deserve investigation.
The strong quantitative agreement with $\Lambda$CDM (Section~\ref{sec:results}) suggests this direction warrants further theoretical development, particularly in constructing the full relativistic framework and testing against additional observational constraints.

\section{Predictions and Falsifiability}
\label{sec:predictions}
The External-Node Model makes distinct predictions that differ from standard $\Lambda$CDM cosmology, offering pathways for falsification.
These predictions are not independent: they all emerge from the same discrete lattice geometry that produces the accelerated expansion, with no additional parameters.
Because the virialized meta-structure is irregular (varying node masses and positions), the lattice imprints a hierarchy of angular signatures---dipole, quadrupole, octopole---all sharing a common preferred direction.
Confirmation or falsification of any one prediction constrains the others.

\subsection{The Geometric ``Big Rip'' (Hyper-Acceleration)}
\label{sec:phantom}
Standard $\Lambda$CDM predicts a constant dark energy density, which leads to a smooth exponential expansion ($w = -1$) forever.
The HMEA model predicts a dynamic future.

\paragraph{Prediction:} As the bubble radius $R(t)$ approaches the grid spacing $S$, the tidal force scales as $(S-R)^{-2}$.
The acceleration will not remain constant; it will increase as the leading edge of the universe approaches the gravity wells of the HMEAs.

\paragraph{Observable:} In deep future epochs, or perhaps measurable in current high-z supernova data, the effective equation of state $w$ should drift below $-1$ (phantom energy behavior \cite{Caldwell2003}), indicating stronger-than-exponential acceleration as we approach the node boundary.

\paragraph{Quantitative Analysis:}
We compute the total effective equation of state parameter using $w_{eff} = -1 - \frac{2}{3}\frac{d \ln H}{d \ln a}$, which characterizes the net deceleration parameter of the expansion rather than the dark energy component alone.
For the balanced-optimization configuration M = 9000 $\times$ M$_{obs}$, S = 38 Gpc, the current universe (RMS radius $\sim$7 Gpc) has $R/S \approx 0.19$---well within the linear tidal regime where the model closely tracks $\Lambda$CDM.

The phantom deviation grows as $R/S$ increases toward unity, with the tidal force scaling as $(S - R)^{-2}$.
Configurations with smaller $S$ (e.g., S = 15 Gpc, $R/S \approx 0.48$) would exhibit stronger present-day phantom signatures.
This provides a quantitative test: future precision measurements of $w(z)$ at low redshift should detect systematic drift below $w = -1$ if external-node dynamics are correct, with the magnitude depending on the true $R/S$ ratio.

\subsection{Dipole Anisotropy}
\label{sec:dipole}
While the Progenitor Hypothesis suggests a roughly isotropic start, the virialized (irregular) nature of the meta-grid implies the nearest HMEAs likely have different masses or distances.

\paragraph{Prediction:} Deep-field surveys (e.g., Euclid \cite{Euclid2011}, LSST \cite{LSST2019}) should detect a subtle \textbf{dipole in the expansion rate}.
One hemisphere of the sky (facing the nearest or largest HMEA) should exhibit a slightly higher local Hubble constant ($H_0$) than the opposing hemisphere.
This anisotropy may contribute to the observed ``Hubble Tension''---the $\sim$8.6\% discrepancy between early-universe (CMB-calibrated, $\sim$67 km/s/Mpc) and late-universe (distance ladder, $\sim$73 km/s/Mpc) measurements of $H_0$ \cite{Riess2022, DiValentino2021}.
While the tension is conventionally framed as a disagreement between measurement \textit{methods}, a directional dipole in the true expansion rate could manifest as an apparent method-dependent discrepancy if early- and late-universe probes preferentially sample different sky regions or weight the sky differently.
In particular, CMB-derived $H_0$ is an all-sky average, while distance-ladder measurements are concentrated along specific lines of sight; a $\sim$5-11\% dipole could bias the latter relative to the former.

\paragraph{Quantitative Estimate:}
In a virialized meta-structure, nodes exhibit both \textbf{position irregularity} (deviations from perfect lattice) and \textbf{mass variation} (nodes have different masses).
Both contribute to asymmetric tidal fields. We estimate the resulting $H_0$ dipole analytically, assuming 5\% position irregularity and 20\% mass variation---conservative values for a virialized structure.
We define the effective dark energy density parameter $\Omega_{\Lambda,eff} \equiv 2 G M_{ext} / (S^3 H_0^2)$, which equals $\Omega_\Lambda$ when the tidal acceleration matches the cosmological constant (Equation~\ref{eq:matching}).
The tidal acceleration asymmetry propagates to $H_0$ via:

\[
\frac{\Delta H_0}{H_0} \approx \frac{1}{2} \cdot \frac{\Omega_{\Lambda,eff}}{\Omega_m + \Omega_{\Lambda,eff}} \cdot \sqrt{\left(\frac{2\delta_{eff}}{S - R}\right)^2 + \left(\frac{\Delta M_{eff}}{M}\right)^2}
\]

where position and mass variations add in quadrature (uncorrelated).
Here $\delta_{eff} = \delta_{pos}/\sqrt{6}$ and $\Delta M_{eff} = \Delta M / \sqrt{6}$ account for statistical cancellation across the 6 nearest face nodes, with $\delta_{pos}$ and $\Delta M$ denoting the per-node position irregularity and mass variation respectively.
For M = 9000 $\times$ M$_{obs}$, S = 38 Gpc, $R \approx 7.25$ Gpc (today):

\begin{itemize}
    \item Position irregularity only (5\%): $\Delta H_0 / H_0 \approx 2.4\%$ (grid average)
    \item Mass variation only (20\%): $\Delta H_0 / H_0 \approx 3.9\%$ (grid average)
    \item \textbf{Combined (position + mass)}: $\Delta H_0 / H_0 \approx 4.6\%$ ($\sim$3.2 km/s/Mpc, grid average)
    \item \textbf{Single nearest node (worst case)}: $\Delta H_0 / H_0 \approx 11.3\%$ ($\sim$7.9 km/s/Mpc)
    \item Observed Hubble Tension: $\sim$8.6\% (6 km/s/Mpc difference between 67 and 73 km/s/Mpc)
\end{itemize}

\paragraph{Implications:}
The predicted dipole ($\sim$4.6\% grid average, up to $\sim$11.3\% worst-case) is comparable to the Hubble Tension ($\sim$8.6\%).
This provides a falsifiable prediction with significant discriminating power: if high-precision all-sky $H_0$ measurements (e.g., from Euclid, LSST) detect a dipole anisotropy at the $\sim$5-11\% level aligned with no known local structure, it would strongly support the external-node hypothesis.
Conversely, the absence of any dipole at the $>$2\% level would constrain the model's parameter space, requiring either more symmetric node distributions or smaller mass variations than expected for virialized structures.

\subsection{CMB Large-Angle Anomalies (The ``Axis of Evil'')}
\label{sec:axis-of-evil}

The CMB's lowest multipole moments exhibit an anomalous alignment known as the ``Axis of Evil'' \cite{Land2005}: the quadrupole ($l=2$) and octopole ($l=3$) share a common preferred axis and are mutually aligned at a level inconsistent with statistical isotropy ($p < 0.5\%$) \cite{deOliveiraCosta2004, Copi2006}.
This axis also correlates with the ecliptic plane and the CMB dipole direction.
Standard $\Lambda$CDM, which predicts statistically isotropic fluctuations with uncorrelated multipole orientations, offers no explanation for this alignment.

\paragraph{Prediction:}
The External-Node Model provides a natural geometric origin for large-angle CMB anomalies.
The 3$\times$3$\times$3 HMEA lattice contains nodes at three distinct distances: 6 face nodes at distance $S$, 12 edge nodes at $S\sqrt{2}$, and 8 corner nodes at $S\sqrt{3}$.
This discrete geometry imprints a hierarchy of angular modes on the tidal field:

\begin{itemize}
    \item The \textbf{dipole} ($l=1$) arises from the nearest node being slightly closer or more massive than the opposing node (as quantified in Section~\ref{sec:dipole}).
    \item The \textbf{quadrupole} ($l=2$) arises naturally from the cubic lattice geometry: face nodes along the three principal axes create a tidal field with three-fold directional structure. Any mass or distance asymmetry between the $\pm x$, $\pm y$, $\pm z$ face-node pairs generates quadrupolar anisotropy.
    \item The \textbf{octopole} ($l=3$) receives contributions from the 8 corner nodes at $S\sqrt{3}$, whose tetrahedral sub-symmetry can produce $l=3$ angular structure.
\end{itemize}

Because all multipoles originate from the same lattice, their preferred axes are necessarily correlated---this is precisely the alignment observed in the CMB data.
The Progenitor Hypothesis (Section~\ref{sec:progenitor}) reinforces this: matter expanding from a destabilized node inherits the tidal imprint of the surrounding lattice from the earliest moments.
The observed correlation of the Axis of Evil with the ecliptic plane is likely coincidental and should not be attributed to external-node dynamics without further investigation.

\paragraph{Qualitative Comparison:}
The observed anomalies include (1) anomalously low quadrupole power, (2) planarity of the octopole, and (3) mutual alignment of $l=2$ and $l=3$ with each other and with a preferred sky direction \cite{Copi2010}.
The External-Node Model qualitatively accounts for all three: the quadrupole is suppressed because the cubic lattice's approximate symmetry partially cancels the $l=2$ mode (only asymmetries between opposing face nodes contribute), the octopole is planar because corner-node contributions project onto the lattice's principal planes, and the alignment is guaranteed because both modes derive from the same physical structure.

\paragraph{Testable Implication:}
If the Axis of Evil is caused by the HMEA lattice, the predicted $H_0$ dipole (Section~\ref{sec:dipole}) should be \textbf{aligned with the CMB quadrupole-octopole axis}.
This provides a cross-check: future all-sky $H_0$ measurements (Euclid, LSST) finding a dipole aligned with the known Axis of Evil direction ($l \approx 260°$, $b \approx 60°$ in galactic coordinates) would strongly support the external-node hypothesis.
Conversely, a dipole orthogonal to the Axis of Evil would disfavor a single-lattice origin for both anomalies.

\subsection{Dark Flow and Large-Scale Bulk Motions}
\label{sec:dark-flow}

Observations of galaxy cluster peculiar velocities reveal anomalously large coherent bulk flows on scales where $\Lambda$CDM predicts they should have decayed to near zero.
Kashlinsky et al. \cite{Kashlinsky2008} first reported a ``dark flow'' of $\sim$600-1000 km/s extending to $\sim$800 Mpc using the kinematic Sunyaev-Zeldovich effect on WMAP data.
While the original kSZ detection remains disputed \cite{PlanckKSZ2014}, independent measurements using galaxy distance indicators have strengthened the anomaly:
Watkins et al. \cite{Watkins2023} measured a bulk flow of $419 \pm 36$ km/s at $\sim$290 Mpc using the CosmicFlows-4 catalog, finding 4.8$\sigma$ tension with $\Lambda$CDM expectations.
Critically, follow-up analysis \cite{Watkins2025} demonstrated that this flow is dominated by \textbf{gravitational sources external to the survey volume}---precisely the signature expected from trans-observable structure.

\paragraph{Prediction:}
The External-Node Model provides a direct mechanism for dark flow.
Any asymmetry in the HMEA lattice (the nearest node being closer, more massive, or both) produces a net gravitational acceleration on the entire observable universe.
Unlike locally-sourced peculiar velocities, which diminish at larger scales as overdensities average out, this externally-sourced flow is \textbf{scale-independent}: every object in the bubble experiences the same tidal bias, regardless of distance from us.

\paragraph{Quantitative Consistency:}
The predicted $H_0$ dipole of $\Delta H_0/H_0 \approx 4.6\%$ (Section~\ref{sec:dipole}) corresponds to an asymmetric acceleration that, integrated over cosmic time, produces bulk velocities of order:
\[
v_{bulk} \sim \frac{\Delta H_0}{H_0} \times H_0 \times R_{eff} \approx 0.046 \times 70 \text{ km/s/Mpc} \times 100 \text{ Mpc} \approx 320 \text{ km/s}
\]
where $R_{eff} \sim 100$ Mpc is a characteristic measurement scale, chosen to approximate the effective depth of peculiar velocity surveys such as CosmicFlows-4 \cite{Watkins2023}. This is an order-of-magnitude estimate; the precise bulk velocity depends on the integration of asymmetric tidal forces over cosmic time.
This is consistent with the observed $\sim$400 km/s---reasonable given the order-of-magnitude nature of this analytical estimate. The single nearest-node scenario ($\Delta H_0/H_0 \approx 11.3\%$) would yield $\sim$790 km/s, bracketing the observed value.

\paragraph{Directional Coherence:}
The dark flow direction ($l \approx 290°$, $b \approx 30°$ from Kashlinsky; $l \approx 298°$, $b \approx -8°$ from Watkins) is in the same general region of the sky as the CMB Axis of Evil ($l \approx 260°$, $b \approx 60°$) and the CMB dipole.
This approximate concordance across independent methods (kSZ, distance indicators, CMB multipoles) is expected if all three signals trace the same nearest HMEA.
Precise alignment is not expected because the real meta-structure is irregular.

\section{Observational Consistency and Defenses}
\label{sec:defenses}
The HMEA model faces two immediate observational challenges: the invisibility of the attractors (why don't we see them?) and the apparent isolation of our universe.
We propose that these are not contradictions, but necessary consequences of the meta-structure's extreme age and scale.
\subsection{The ``Dark Giant'' Principle (Spectral Quiescence)}
\label{sec:dark-giant}
\paragraph{Critique:} Even if an object is 30 Gpc away, a singular mass of $10^{55}$ kg should surely emit detectable radiation, perhaps from an accretion disk or Hawking radiation.
Why is the sky dark in those directions?

\paragraph{Defense:} The most parsimonious explanation is that HMEAs are ancient black holes that have long since exhausted their local fuel supply.

\paragraph{Mechanism:} An HMEA is vastly older than the current expansion bubble ($t_{node} \gg 13.8 \text{ Gyr}$)—possibly quadrillions of years old.
Over such timescales, it has accreted all available matter in its immediate vicinity, effectively clearing its ``cell'' in the grid.

\paragraph{Result:} Without an active accretion disk to generate thermal radiation or relativistic jets, a black hole is electromagnetically silent.
It is a ``naked mass'' in the meta-structure—detectable only through its gravitational gradient (Dark Energy), not its luminosity.
Furthermore, Hawking radiation is negligible at these masses: the Hawking temperature scales as $T_H \propto 1/M$, yielding $T_H \sim 10^{-83}$ K for $M \sim 10^{55}$ kg—effectively zero at the source, before any propagation losses are considered.

\subsection{Fossil Black Holes and Dark Matter}
\label{sec:fossil}
\textit{(Speculative---no simulation support)}

While the nodes themselves are invisible, the HMEA framework predicts that the void space between the bubble and the nodes is not perfectly empty.
It may contain remnant debris from previous epochs of the meta-structure's evolution.
In particular, the unexplained population of intermediate-mass black holes (IMBHs) (wandering objects too massive to be stellar remnants yet too small to be galactic nuclei) may offer an alternative explanation as fossil remnants.
These could be debris from the HMEA's previous accretion cycles, ancient compact objects lingering in the void until encompassed by our expansion.
While highly speculative, such objects might even contribute to the observed dark matter budget if sufficiently numerous and small-scale.
This remains an open question requiring further theoretical development and observational constraints.

\section{The ``Great Metabolism'' Hypothesis}
\label{sec:metabolism}
\textit{(Speculative Extension---not derived from simulation results)}

The HMEA framework implies a cyclical, evolutionary cosmology that radically reframes our place in the universe.
We propose that cosmic expansion is not a linear march toward entropy death, but a metabolic phase of the meta-structure.

\begin{enumerate}
    \item \textbf{Expansion/Accretion}: An "explosion" (Big Bang) fills a void with matter.
As the bubble expands, it acts effectively as a macroscopic accretion disk, transporting matter across the void.
\item \textbf{Feeding}: Our galaxies are eventually stripped away and accreted onto the HMEAs, adding to their mass and energy.
The universe is "consumed."
    \item \textbf{Destabilization}: After eons of growth, an HMEA accumulates critical mass or destabilizes due to orbital decay and merger with a neighbor node.
\item \textbf{Re-Ignition}: The node detonates, redistributing its mass into the void and initiating a new Big Bang cycle.
\end{enumerate}

In this view, our universe is a transient structure—a pulse of energy and matter—facilitating the transfer of mass between nodes in an eternal cosmic lattice.
We are the fuel for the next generation of attractors.

\section{Conclusion}
\label{sec:conclusion}
The External-Node Model offers a geometric solution to the Dark Energy problem.
By reframing cosmic acceleration as the consequence of gravitational attraction from a larger, trans-observable manifold, we eliminate the need for fine-tuned vacuum energy and the associated theoretical crises of the Standard Model.

Through N-body simulations with 2000 particles, we have demonstrated that multiple parameter configurations across a wide mass range (M = 92-9000 $\times$ M$_{obs}$, S = 15-38 Gpc) reproduce $\Lambda$CDM expansion dynamics with $R^2 > 0.99$ for size evolution and $R^2 > 0.95$ for expansion rate over 8 Gyr ($t = 5.8 \rightarrow 13.8$ Gyr).
Size-only optimization can push the size $R^2$ to 0.9991, though at the cost of expansion rate fidelity---a tradeoff inherent to optimizing an integrated versus derivative quantity (Section~\ref{sec:tradeoffs}).
This validates the core mechanism: external gravitational sources can mimic dark energy using only classical physics.

Beyond the expansion match, the same lattice geometry---with no additional parameters---generates predictions consistent with three independent large-scale anomalies: the Hubble Tension ($\Delta H_0/H_0 \approx 5$-$11\%$), the CMB Axis of Evil (correlated quadrupole-octopole alignment), and dark flow ($\sim$320-790 km/s bulk motions).
These are typically treated as unrelated curiosities; in the External-Node Model, they are necessary consequences of a single geometric structure sharing a common preferred direction (Section~\ref{sec:predictions}).
While each connection remains qualitative and requires quantitative validation, the convergence of multiple independent lines of evidence from a parameter-free geometric prediction is suggestive.

While this work presents a simplified toy model with acknowledged limitations, particularly regarding early universe physics and the need for full relativistic treatment, it demonstrates several important principles.

First, \textbf{dark energy is not inevitable}: classical gravitational alternatives can reproduce the observed acceleration without invoking vacuum energy.

Second, \textbf{fundamental assumptions may break down}: the Cosmological Principle might not extend to super-horizon scales, opening new theoretical possibilities.

Third, \textbf{the mechanism is unifying}: a single geometric structure can simultaneously address the expansion history and multiple apparently unrelated anomalies.

Finally, \textbf{new research directions emerge}: even if this specific model is ultimately superseded, it may inspire more complete theories that challenge orthodox assumptions.

Future work must focus on extending the framework to address early universe physics, computing the quantitative CMB multipole predictions from the lattice geometry, developing the full general relativistic formulation, and testing directional coherence between the predicted $H_0$ dipole, CMB multipole alignment, and bulk flow vector using upcoming survey data (Euclid, LSST).
While the existence of trans-observable structure remains speculative, the convergence between this toy model's geometric predictions and multiple independent observational anomalies suggests that alternatives to dark energy deserve serious theoretical investigation.

\begin{thebibliography}{99}

\bibitem{Planck2020} Planck Collaboration. "Planck 2018 results. VI. Cosmological parameters." \textit{Astronomy \& Astrophysics} 641, A6 (2020).

\bibitem{Weinberg1989} Weinberg, S. "The cosmological constant problem." \textit{Reviews of Modern Physics} 61.1: 1 (1989).

\bibitem{Steinhardt1999} Zlatev, I., Wang, L., and Steinhardt, P. J. "Quintessence, cosmic coincidence, and the cosmological constant." \textit{Physical Review Letters} 82.5: 896 (1999).

\bibitem{Riess2022} Riess, A. G., et al. "A Comprehensive Measurement of the Local Value of the Hubble Constant with 1 km/s/Mpc Uncertainty." \textit{The Astrophysical Journal Letters} 934.1: L7 (2022).

\bibitem{Land2005} Land, K. and Magueijo, J. "Examination of evidence for a preferred axis in the cosmic radiation anisotropy." \textit{Physical Review Letters} 95.7: 071301 (2005).

\bibitem{Watkins2023} Watkins, R., et al. "Analysing the large-scale bulk flow using CosmicFlows4." \textit{Monthly Notices of the Royal Astronomical Society} 524.2: 1885 (2023).

\bibitem{Riess1998} Riess, A. G., et al. "Observational evidence from supernovae for an accelerating universe and a cosmological constant." \textit{The Astronomical Journal} 116.3: 1009 (1998).

\bibitem{Perlmutter1999} Perlmutter, S., et al. "Measurements of $\Omega$ and $\Lambda$ from 42 high-redshift supernovae." \textit{The Astrophysical Journal} 517.2: 565 (1999).

\bibitem{MersiniHoughton2009} Mersini-Houghton, L. and Holman, R. "``Tilting'' the universe with the landscape multiverse: the dark flow." \textit{Journal of Cosmology and Astroparticle Physics} 2009.02: 006 (2009).

\bibitem{Kashlinsky2008} Kashlinsky, A., et al. "A measurement of large-scale peculiar velocities of clusters of galaxies: results and cosmological implications." \textit{The Astrophysical Journal} 686.2: L49 (2008).

\bibitem{Birkhoff1923} Birkhoff, G. D. \textit{Relativity and Modern Physics}. Harvard University Press, 1923.

\bibitem{Eisenstein2005} Eisenstein, D. J., et al. "Detection of the baryon acoustic peak in the large-scale correlation function of SDSS luminous red galaxies." \textit{The Astrophysical Journal} 633.2: 560 (2005).

\bibitem{Caldwell2003} Caldwell, R. R., Kamionkowski, M., and Weinberg, N. N. "Phantom energy and cosmic doomsday." \textit{Physical Review Letters} 91.7: 071301 (2003).

\bibitem{Euclid2011} Laureijs, R., et al. "Euclid Definition Study Report." arXiv preprint arXiv:1110.3193 (2011).

\bibitem{LSST2019} Ivezić, Ž., et al. "LSST: from science drivers to reference design and anticipated data products." \textit{The Astrophysical Journal} 873.2: 111 (2019).

\bibitem{DiValentino2021} Di Valentino, E., et al. "In the realm of the Hubble tension—a review of solutions." \textit{Classical and Quantum Gravity} 38.15: 153001 (2021).

\bibitem{deOliveiraCosta2004} de Oliveira-Costa, A., et al. "Significance of the largest scale CMB fluctuations in WMAP." \textit{Physical Review D} 69.6: 063516 (2004).

\bibitem{Copi2006} Copi, C. J., et al. "On the large-angle anomalies of the microwave sky." \textit{Monthly Notices of the Royal Astronomical Society} 367.1: 79-102 (2006).

\bibitem{Copi2010} Copi, C. J., et al. "Large-angle anomalies in the CMB." \textit{Advances in Astronomy} 2010: 847541 (2010).

\bibitem{PlanckKSZ2014} Planck Collaboration. "Planck intermediate results. XIII. Constraints on peculiar velocities." \textit{Astronomy \& Astrophysics} 561: A97 (2014).

\bibitem{Watkins2025} Watkins, R. and Feldman, H. A. "The bulk flow of galaxy surveys." arXiv preprint arXiv:2512.03168 (2025).

\end{thebibliography}

\end{document}